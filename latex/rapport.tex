\documentclass[12pt]{report}
%% Language and font encodings
\usepackage[francais]{babel}
\usepackage[utf8]{inputenc}
\usepackage[T1]{fontenc}
\usepackage{lmodern} 
\usepackage[svgnames]{xcolor}
\usepackage{float} % figure
\usepackage{eurosym} % euro character
\usepackage{minted} % syntax coloring
\usepackage{amsmath}
\usepackage{graphicx}
\usepackage[colorlinks=true, allcolors=blue]{hyperref}

%% Sets page size and margins
\usepackage[a4paper,top=3cm,bottom=2cm,left=3cm,right=3cm,marginparwidth=1.75cm]{geometry}

%% Syntax color for minted code blocks
\usemintedstyle{tango}

\title{Licence ADSILLH 2017/2018\\Rapport de projet Réseau:\\Pong
  multijoueur en Python}
\author{Pierre Antoine Rouby - David Tabarie\newline}

\date{}

\begin{document}
\maketitle

\begin{abstract}
\end{abstract}
\tableofcontents

\part{Présentation du projet}

\part{Idée de départ}
\chapter{Architecture du programme}
Le programme doit être un pong en réseaux et doit être jouable à 2;
plusieurs solution de connections réseaux s'offrent donc à nous:
\begin{itemize}
\item Un système Client/Serveur centralisé: \\
  Un tel système nécessite un serveur central ainsi que deux clients, ceux-ci se
  connectent au server et créent une partie privée via une clé de
  session que seul les deux joueurs pourront rejoindre, le serveur
  central doit donc pouvoir gérer plusieurs parties en même temps. \\
  Le problème de cette solution et qu'il faut entretenir un serveur 24h/24
  car sans lui les joueurs ne pourrons pas jouer.
  Même si on partage les sources du serveur tous les joueurs n'ont pas forcément
  envie ou les moyens de mettre en place un server pour une partie ou deux.
\item Un système Client/Serveur acentré: \\
  Ce système et basé sur un même fichier exécutable qui va pouvoir prendre
  soit le rôle de serveur soit de client. l'avantage de ce système et qu'il n'y
  a pas besoins de lancer un autre programme, il faut juste que les
  joueurs décident de qui va se connecter à qui. \\
  \textit{Un problème peut se poser si les deux joueur ne sont pas sur le même
    réseau et que les ports ne sont pas ouvert. Des solutions doivent exister
    mais nous n'avons pas eu le temps de chercher}
  Se système se rapproche d'un P2P (peer2peer) car le server et le client
  aurons peu ou prou le même rôle, chacun devra vérifier que l'autre ne triche
  pas. La principale différence est que le client va prendre les paramètres de
  jeux du serveur à sa connection.
  Ici le serveur n'a pas d'autorité spécifique, les deux côtés doivent être au
  courant de la position de l'autre.
  \textit{Ce qui peut poser problème en cas de ping élevé}
\end{itemize}
Nous avons optés pour un système Client/Serveur acentré avec une option au
lancement du programme \textit{main.py --server} ou \textit{main.py --client}.
Nous avons choisi une connection en \texttt{tcp} pour pouvoir repérer plus
facilement les problèmes de déconnections (\textit{rage quit}).

\chapter{Game design}
Pour le design du pong nous avons choisi de voir les deux joueurs sur l'écran
du même coté de l'écran pour ne pas avoir une fenêtre 2 fois plus grande,
la raquette du deuxième joueur doit apparaître en opacité inférieure pour que
le premier comprenne instinctivement que ce n'est pas sa raquette, et inversement
sur l'écran du deuxième.
De plus il faut que lorsque la balle rebondie sur le mur d'en face son opacité
change pour signifier au joueur que c'est à lui ou à son adversaire de toucher la
balle.
Le fait de voir les deux raquettes permet aussi de repérer une triche éventuelle.

Nous avions aussi prévus un panel de thèmes diffèrents pour la
raquette, la balle et le fond d'écran:
\begin{itemize}
\item Star Wars
\item Doom
\item Kung fury
\item etc.
\end{itemize}
Avec en plus de la possibilité pour le joueur de faire des thèmes personnalisés
avec ses propre images et musique.

\part{Recherche de solutions}
\chapter{Définition d'un protocole}
Il a donc fallu définir un protocole de communication qui permet de communiquer
pour l'échange d'information.
Pour cela nous somme parti de l'idée d'une commande accompagnée d'argument sous la
forme : \texttt{COMMANDE:arg1,arg2,arg3;}.

La commande est séparée des arguments par deux point ':' et chaque argument est
séparé par une virgule ',' et la commande de termine par un point virgule ';'.

Pour simplifier la programation de la réception de commande il fut convenu
que la commande ne dépasserai pas 4 caractères (MOVE / CONN). Ainsi il est
possible de récupérer séparément la commande et les arguments comme ceci:
\begin{minted}{python}
def parse_cmd(self, cmd):
   act = cmd[:4]
   if cmd[-1:] == ';':
      arg = cmd[5:-1]
   else:
      arg = cmd[5:]
   sarg = arg.split(",")
   return act, sarg
\end{minted}

Pour récupérer l'action demandée, il suffit de noter que l'on veut les quatre premiers
caractères de la chaîne, cette méthode est sans doute moins coûteuse qu'un split
au niveau des deux points (':').

A noté qu'ici on peut voire une condition sur le dernier caractère de la chaîne
cmd, elle permet de corrigé un bug que nous avons rencontré car la récupération
des commande par le réseau peut se faire de deux façons différente.
Le fait qu'il y ai deux façons de récupérer les commandes vient lui même d'un
problème rencontré lors de la lecture des données du réseau. En effet
plusieurs commandes peuvent être lues en même temps sur le réseaux, il faut
donc découper la chaîne des données brute reçuent au niveau du point virgule
(';').
Le problème est que la séquence de connections et d'échange de paramètre doit
respecter un ordre précis et donc nous utilisons une autre fonction qui est
faite récupéré un seul commande à la fois et la valeur retourné par cette
fonction se termine par ';'.
% Pas compris dernière phrase

\chapter{Choix concernant l'affichage graphique} 
Pour des raisons de modularité, nous avons décidés d'employer le
paradigme objet, celui-ci se trouve en effet particulièrement bien
adaptés aux jeux vidéos et autres applications graphiques. Le but
était de faciliter les ajouts d'éléments futurs...

\part{Mise en place des solutions}
\chapter{Architecture du projet}
\begin{itemize}
\item[main.py: ] Programme exécutable : 
\item[sock.py: ] Class Sock: Ouverture de connection réseau et envoi
  et reception de données 
\item[protocol.py: ] Class Protocol : Protocole de communication pour le
  multi-joueur
\item[pong.py: ] Class Ball / Class Paddle / Class Game : Éléments graphiques
  pour le jeux de pong 
\end{itemize}

\chapter{Mise en place du protocole}
Pour la mise en place du protocole pour la communication en multi-player le
problème principal est de ne pas ralentir le programme ou ne pas le bloquer.
La première solution à laquelle nous avons pensé et de multi-threader le
programme avec un thread de réception des données sur le réseau.

Cette solution s'est révélée trop gourmande en CPU et demande à utiliser des
mécanismes d'exclusion étant donné que les méthodes \texttt{recv} de la classe
\texttt{Socket} ne sont pas \texttt{Thread Safety}.
L'utilisation de mutex rend plus difficile l'écriture du programme et l'on
prend le risque d'un \textit{deadlock}.

Nous avons donc changés de solution pour faire la lecture des données reçu
par le réseau dans la boucle principale.
Un problème s'est posé car par défaut le comportement de la
méthode \textit{socket.recv} est d'attendre que des données soient reçues (la
méthode est dite blockent). Il faut donc demander à ce que la méthode
n'attende pas de données et récupère uniquement les données déjà
disponibles ou si il n'y a rien retourne à la fonction appellante.
Pour cela nous avons définis la méthode :
\begin{minted}{python}
  def set_recv_no_blocking(self):
    self.data.setblocking(0)
\end{minted}

Il a fallu aussi changer la méthode de réception des données comme ceci:
\begin{minted}{python}
  try:
    answer = self.data.recv(BUF_SIZE)
    return answer.decode()
  except:
    return ['']
\end{minted}

Nous avons dû utiliser un test de type Try/Except car si aucune donnée n'est lue
lors du \textit{recv} une exécution a lieu et quitte le programme si elle
n'est pas gérée.
\textit{Cette version de la méthode et simplifié mais surtout imparfaite
  car nous ne testons pas quelle exception a eu lieu.}

Nous avons pensé à d'autre commandes pour le multijoueur, comme \texttt{SYNC} qui
devait permettre de synchroniser l'état et la position du client et du server.
Nous pensions l'envoyer à intervale régulier via un signal \texttt{ALRME} (alarme),
le problème et qu'avec le ping et le fait que le message sera lu dans un laps
de temps allant de 0 a 10ms (dû au delay) nous risquions d'augmenter la
dé-synchronisation plus qu'autre chose, nous avons donc décidés de ne
pas l'intégrer.

Au final nous avons remarquer ne nombreuse désyncronisation, nous avons donc
mis en place des commandes (\texttt{TOUC / THRW}) de syncronisation sur des
evenements.
C'est commande sont envoyer avec une disaine d'arguement se qui permet de toujours
avoir un états très similaire sur le client et le server.

\chapter{Mise en place de l'interface graphique}
Nous nous sommes dans un premier temps basés sur l'exemple de code
fournis, puis l'avons adapté afin qu'il soit orienté objet. Nous avons
regroupés tous les éléments graphiques dans un fichier, celui-ci
contient les classes suivantes:

\begin{description}
  \item [Paddle:] elle définit les object de type ``raquettes'' du jeu, elle
    comprend notamment les méthodes move_paddle et draw.
  \item [Ball:] contrairement à la balle d'origine qui était un
    sprite, cette classe génère un cercle au comportement similaire;
    l'avantage est qu'il est plus aisé de changer la couleur.
  \item [Item: ] après hésitation, nous avons décidés de laisser dans
    le code cette classe inachevée, elle était vouée à servir pour
    instancier des objets bonus.
  \item [Score: ] ajoute en haut à droite de l'écran un compteur de
    points.
  \item [Game: ] classe la plus conséquente du fichier,
    contrairement à de nombreux jeu et logiciels écrits en paradigme
    objet, nôtre classe Game ne contient pas la boucle principale du
    programme; cela est dû à la contrainte du multijoueur, ainsi, Game
    est instancié dans le fichier main.py .
\end{description}

Nous avons aussi rencontré des problème de syncronisation entre le client
et le server, et cêla même si il sont tous deux sur la même machine et on
donc un ping très faible.
Pour réduire la désincronisation et améliré la stabilité du ``framrate''
nous avons ajouté un ``timer'' qui calcule le temps que met le jeux à calculer
de different élement, récupéré les donnée, et mettre à jour l'affichage et le
soutré au temps d'attente prévu de 10 ms.

\chapter{Mise en commun}
Le dévelopement s'étant déroulé à distance, nous avons eu quelques
difficultés à nous organiser, aussi une des difficulté majeure fut de
mettre en commun des morceaux de code disparates.
En effet nous avois juger qu'il n'étais néssesaire de faire des
schema, se qui a un pour conséquance des fonctionnement très different
des deux côtés.
Mais une meilleur préparation aurai sans dout put nous
permettre de gagner du temps avec un code au final plus coérrent. 

Cela nous a aussi forcé à revenir sur certains points techniques,
ainsi que questionner certaines des fonctionnalités que nous voulions
implémenter.

Un autre problème a étais notre gestion du temps, nous n'avons pas
suffisament avancer au debut du développemet, se qui à eu pour
conséquance de ne pas pouvoir réalisé toutes les fonctinnalité prévu.
\end{document}
