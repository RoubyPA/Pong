\documentclass[12pt]{report}
%% Language and font encodings
\usepackage[francais]{babel}
\usepackage[utf8]{inputenc}
\usepackage[T1]{fontenc}
\usepackage{lmodern} 
\usepackage[svgnames]{xcolor}
\usepackage{float} % figure
\usepackage{eurosym} % euro character
\usepackage{minted} % syntax coloring
\usepackage{amsmath}
\usepackage{graphicx}
\usepackage[colorlinks=true, allcolors=blue]{hyperref}

%% Sets page size and margins
\usepackage[a4paper,top=3cm,bottom=2cm,left=3cm,right=3cm,marginparwidth=1.75cm]{geometry}

%% Syntax color for minted code blocks
\usemintedstyle{tango}

\title{Licence ADSILLH 2017/2018\\Rapport de projet Réseau:\\Pong
  multijoueur en Python}
\author{Pierre Antoine Rouby - David Tabarie\newline}

\date{}

\begin{document}
\maketitle

\begin{abstract}
\end{abstract}
\tableofcontents

% Je sais pas du tout si la classe report est une bonne idée...
\part{Présentation du projet}
% Reformuler le pdf

\part{Recherche de solutions}
\chapter{Définition d'un protocole}

\chapter{Choix concernant l'affichage graphique} % Titre moyen
Pour des raisons de modularité, nous avons décidés d'employer le
paradigme objet, celui-ci se trouve en effet particulièrement bien
adaptés aux jeux vidéos. Le but était de faciliter les ajouts
d'éléments futurs...

\part{Mise en place des solutions}
\chapter{Mise en place du protocole}

\chapter{Mise en place de l'interface graphique}

\chapter{Mise en commun}
% Là on peut mettre qu'on en a chié parce qu'on a pas assez schématisé,
% je pense que si on montre qu'on a repéré nos erreurs ça peut être
% un bon point. (Oui/Non ?)

\end{document}
